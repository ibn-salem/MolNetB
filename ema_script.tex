\documentclass{llncs}
\usepackage{amsmath,amsfonts,amssymb}   %math. Symbole und Umgebungen
\bibliographystyle{splncs}

\title{Designing optimal cell factories:
integer programming couples elementary mode analysis with regulation}
\author{Jonas Ibn-Salem}
\institute{}%Freie Universität Berlin}

\begin{document}

\maketitle

\begin{abstract}
This is an extended abstract of a recent publication \cite{Jungreuthmayer2012}
prepared for the Seminar "Molecular Networks B" SS 2013 supervised by 
Prof. Dr. Alexander Bockmayr at the Free University Berlin.
\end{abstract}
%%%%%%%%%%%%%%%%%%%%%%%%%%%%%%%%%%%%%%%%%%%%%%%%%%%%%%%%%%%%%%%%%%%%%%%%
\section{Background and introduction}
%%%%%%%%%%%%%%%%%%%%%%%%%%%%%%%%%%%%%%%%%%%%%%%%%%%%%%%%%%%%%%%%%%%%%%%%
%In metabolic engineering microbes can be used as production organism for 
%the efficient utilization and production of metabolites.
%Optimization of metabolic processes in living organism can be modelled by
%constraint based methods.
The rapidly growing body of molecular knowledge can be used to optimize
metabolic engineering of living organisms 
like the production of target metabolites by microbes.
Such an optimization can be realized by deactivating unwanted pathways 
through experimental gene deletions to design a strain of minimal functionality 
that efficiently fulfill only the desired functions.
 
Metabolic processes are formalized in metabolic networks for many organisms.
Such a networks with $n$ reactions and $m$ internal metabolites
is usually modelled by a stoichiometric matrix $S \in \bbbr^{m\times n}$, 
a set of irreversible reactions $Irr \subseteq \{1, \dots, n \}$,
and a flux vector $v \in \bbbr^{n}$, where $v_{i}$ describes the flux of
reaction $i$ through the network. 
Reasoned by the law of mass balance, the assumption that the metabolite 
concentrations and reaction rates are constant leads to the steady state 
flux coen $C = \{v \in \bbbr^n | Sv = 0, v_i \geq 0, i \in Irr \}$,
which describes all possible flux vectors in the system.

Metabolic functionality can be caracterized by \emph{elementary flux modes} (EM), 
which correspond to meaningful biological pathways.
For each $v \in \bbbr^{n}$ the support of $v$ is defined as  
$supp(v) = \{i \in \{1, ..., n \} | v_i \neq 0  \}$.
A flux vector $v \in C$ is an EM if $supp(v)$
is minimal w.r.t $\subseteq$ i.e., if there is no $v' \in C, v' \neq 0$ 
with $supp(v') \subsetneq supp(v)$.  
Furthermore $e = e(v)$ is the binary transformation of a given EM $v$, where
$e_{i} := 1$ if $v_{i} \neq  0$ and $e_{i} := 0$ if $v_{i} = 0$.
%$e_{i} := 
%        \begin{cases}
%            1 & \text{if }  v_{i} \neq 0 \\
%            0 & \text{if }  v_{i} = 0 
%        \end{cases}$
Therefore an EM is a minimal and indivisible set 
of reactions that operates under steady state conditions, 
while obeying all (ir-)reversibility constraints.

%%%%%%%%%%%%%%%%%%%%%%%%%%%%%%%%%%%%%%%%%%%%%%%%%%%%%%%%%%%%%%%%%%%%%%%%
\section{Methodology}
%%%%%%%%%%%%%%%%%%%%%%%%%%%%%%%%%%%%%%%%%%%%%%%%%%%%%%%%%%%%%%%%%%%%%%%%
\subsection{Problem definition}
Given all $q = r+s+t$ elementary modes of a metabolic network and a specific design goal,
the EMs can be grouped into three categories.
The \emph{goal} set $\{e_{1}, ..., e_{r}\}$ contains all desirable EM that contribute 
to the design criterion in an optimal way.
The \emph{kill} set $\{e_{r+1}, ..., e_{r+s}\}$ contains all unwanted EM, 
that have to be deleted from the network.
Finally all other EM that are tolerated and may be active or inactive in 
the target design are pooled in the \emph{helper} set $\{e_{r+s+1}, ..., e_{r+s+t}\}$.
%The Problem is now to find a set of reaction deletions with minimal 
%cardinality 
The Problem is now to find a \emph{minimal cut set} (MCS) of reaction deletions 
that inactivates all unwanted EM from the \emph{kill} set and leave the 
EM from the \emph{goal} set active.

\subsection{Binary linear program}
The described problem can be formalized as an binary linear programm (BLP), 
which is an integer linear program with binary variables.
\begin{subequations}
    \begin{align}
        \max \quad  & ~\|x\|    \label{eq:objfunc}\\
        \mbox{s.t.}\quad  & e^{T}_{g} x = \|e_{g}\| &\quad\quad g \in \{1, \dots, r\}   \label{eq:goal} \\
        & e^{T}_{k} x \leq \|e_{k}\|-1     &\quad\quad k \in \{r+1, \dots, r+s\}  \label{eq:kill} \\
        & e^{T}_{h} x \leq \|e_{h}\|       &\quad\quad h \in \{r+s+1, \dots, r+s+t\}  \label{eq:help} \\
        & x = (x_{1}, ..., x_{n})^{T}, &\quad x_{i} \in \{0, 1\} \forall i
        \end{align}
\end{subequations}
The solution vector $x$ is the binary representation of all reactions 
participating in the designed minimal network and indicates whether a reaction
$i$ can carry flux ($x_{i} = 1)$ or needs to be deleted ($x_{i} = 0$).
Therefore the maximization of $\|x\| = \sum_{i=1}^{n} x_{i}$ is equivalent
to minimizing the number of reaction knockouts $\Delta_{\min} = n - \|x\|$.
The constraints in (\ref{eq:goal}) enforce the solution to keep the 
EMs from the \emph{goal} set active. 
Since inactivation of one reaction of an EM is sufficient to delete the
entire EM, the constraints (\ref{eq:kill}) forbid EM from the \emph{kill} set.
The constraints (\ref{eq:help}) are listed only for the (questionable) 
sake of accounting completely for all EM
and do not constrain the system in any way.
 
The BLP may have no or a finite number of solutions.
However, the constraints are relaxed, i.e. EM are shifted from the 
\emph{goal} set to the \emph{helper} or \emph{kill} set,
it is always possible to find at least one solution.
To enumerate all optimal solutions one can add iteratively 
two additional, so called \emph{no-good cut} constraints 
\begin{subequations}
    \begin{align}
        \sum_{i \in B} x_i & \leq |B| -1  \\
%        \text{\emph{\textcolor{blue}{"no-good cut"}}}\\
        \sum_{i \in N} x_i & \geq 1       
    \end{align}
\end{subequations}, 
where $B = \{i |x_{i}^{(j)} = 1\}$ and $N = \{i |x_{i}^{(j)} = 0\}$ to
forbid the $j$-th solution $x^{(j)}$ and solve the BLP again.

\subsection{Experimental feasibility}
Since the gene-enzyme-reaction mapping is not for every reaction, it is 
sometimes not possible to delete reactions from an optimal solution in
in practice. Furthermore it is easier to delete uptake reactions, by simply
removing the substrate from the medium, than internal reaction, for which the 
gene has to be knocked out.
To account for such experimental difficulties, a weight function can be
added to the objective function (\ref{eq:objfunc})
resulting in $\|w^{T}x\|$. 
In general low weight for uptake reactions and high weights for reactions
with missing genetic annotation, will favour the deletion of uptake reactions
and prevent deletion of unannotated ones.

\subsection{Including regulation}
The mapping between genes, enzymes and reactions as well as transcriptional 
regulatory information can be integrated as long as they are formulated 
as static boolean functions. Each gene $g$ is represented by a logical variable 
$y_{g} \in \{0,1\}$ indicating if the gene is active ($y_{g}=1$) or 
deleted ($y_{g}=0$).
Boolean functions can be converted into inequalities and included as 
additional constrains to the BLP.
For example a reaction $R$ may be catalysed by a complex of two proteins encoded
by $G_{a}$ and $G_{b}$. The corresponding Boolean function 
$G_{a} \land G_{b} \mapsto R$ is equivalent to the inequality 
$0 \leq G_{a} + G_{b} - 2R \leq 1$.
The new objective function $\|w_{2}^{T} x \| + \|y\|$ account for 
the combined effect of both, gene and reaction knockouts.

%%%%%%%%%%%%%%%%%%%%%%%%%%%%%%%%%%%%%%%%%%%%%%%%%%%%%%%%%%%%%%%%%%%%%%%%
\section{Results and Discussion}
%%%%%%%%%%%%%%%%%%%%%%%%%%%%%%%%%%%%%%%%%%%%%%%%%%%%%%%%%%%%%%%%%%%%%%%%

\subsection{Evaluation on real data}

The algorithm was applied to a previously published small scale 
metabolic network of \emph{E. coli} under anaerobic conditions to 
optimize the production of ethanol from glucose.
From the entire 5,010 EM in the network, 12 contribute most efficiently
to the ethanol production and where considered as \emph{goal} set. The
remaining 4,998 EM where unwanted and pooled in the \emph{kill} set.
The BLP methods result in 1,048 MCS of which 252 had the minimal number of
seven deletions. 
However 71\% of these solutions are not feasible due to
missing annotations in eat least one reaction.
To account for the experimental feasibility a weight function were applied,
resulting in eight alternative MCS with 7 feasible solutions.
Notably, an experimentally implemented strain with eight knockouts could
be improved with this results.

The analysis of a greater \emph{E.coli} model, without restriction to glucose uptake under
anaerobic conditions, with 429,276 EM results in 55,488 MCS of which 1,440
consists of minimal 11 reaction deletions.
However none of theme are feasible.
Applying a weight function results in a minimum of 12 feasible reaction 
deletions, of which 5 are uptake reactions.
Deleting these 5 uptake reactions by removing the substrate from the medium
recovers the original model with anaerobic growth on glucose.

\subsection{Complexity and runtime}

Since the number of EM explodes combinatorially with system size,
elementery mode analysis is restricted to small scale metabolic 
networks with hundreds of reactions.
The core of the presented method consist of a binary linear program which 
is in general NP-complete.
However efficient comercial and non-commercial solvers are available, 
which are often able to compress redundant constraints and variables 
in a preprocessing step.
The method can therefore handle even larger datasets.
In practice a model with 271 million EM finds successfully 2,304 MCS in 
122 seconds.
However, the computationally most expensive part is the computation of
all EM in the first place.

%%%%%%%%%%%%%%%%%%%%%%%%%%%%%%%%%%%%%%%%%%%%%%%%%%%%%%%%%%%%%%%%%%%%%%%%
\section{Conclusion}
%%%%%%%%%%%%%%%%%%%%%%%%%%%%%%%%%%%%%%%%%%%%%%%%%%%%%%%%%%%%%%%%%%%%%%%%
Binary linear programming can be used for an efficient and easy to implement
method to predict minimal intervention strategies for a strain design
with desirable metabolic capabilities. 
Experimental difficulties as well as regulatory information can easily be
integrated into a single optimization problem by adding a weight function or 
additional variables and constraints.
Evaluation on real data show that the BLP method predicts biologically
meaningful results in a short runtime.
%%%%%%%%%%%%%%%%%%%%%%%%%%%%%%%%%%%%%%%%%%%%%%%%%%%%%%%%%%%%%%%%%%%%%%%%
\bibliography{uni}

\end{document}

